% !BIB program = bibtex
\documentclass{article}
\begin{document}
This is a document to flesh out ideas about the research proposal. \newline
Look up -  "Adapted from Miner, J.T., \& Miner, L.E. (2005). Models of Proposal Planning and Writing (pp. 139). Praegar, Westport: CT."
\begin{enumerate}
    \item General statement: Dominant Trend
    \item Identification of Gap
    \item Significance of Gap
    \item Implications of Gap in local context(?)
    \item Statement of aims
\end{enumerate}
\section{Introduction}
\section{Research Context}
\subsection{What is modelling social interactions?}
Social interactions occur when people interact in a way such that the outcome of the interaction for
one person is dependent not  only on that person's decision but on the decision of the person (or
people) that they are interacting with ('playing against'). Such interactions can also be said to be
'strategic'. There are no shortage of examples of interactions that fit this criteria - from playing
chess, to investing in the stock market.

To expand a little, when you are playing chess whether your
move is a good one or not depends on the moves your opponent intends to make in future. So, you take
this into account - your move depends on what you think you opponent is going to do. However, your
opponent knows this also, so, their move depends on what they think you're going to do. So, really
your move depends on what you think your opponent thinks you're going to do. Of course, this
recursion can be followed ad infinitum, so we need some a formal way to model these interactions -
this is where Game Theory becomes important.

The main paper I'll be citing is \cite{alger2013homo}

\subsubsection{Strategies}
\subsubsection{With preferences}
\subsubsection{Games \& Nash Equilibrium}
\subsection{Social Preferences}
\subsection{Evolution of Social Preferences}
\subsection{Agent Based Modelling}
\subsubsection{Social Preferences in Artificial Agents}
This is where the Peysacovich paper can be mentioned.
\section{Research Questions/Outcomes}
\section{Method}
\section{Timeline}
\bibliography{thesis}
\bibliographystyle{simple}
\end{document}